\documentclass[11pt,a4paper]{article}
	\author{Hu CiGang}
	\title {服务器主机相关操作}
\usepackage{fontspec}
\usepackage{multirow}
\usepackage{enumerate}
\usepackage{listings}
\usepackage{color}

\definecolor{dkgreen}{rgb}{0,0.6,0}
\definecolor{gray}{rgb}{0.5,0.5,0.5}
\definecolor{mauve}{rgb}{0.58,0,0.82}
%\renewcommand{\thefootnote}{\fnsymbol{footnote}}
%\setcounter{footnote}{-1}



\setmainfont{STFangsong}
\begin{document}
%	\maketitle


 
\lstset{ %
  language=Octave,                		% the language of the code
  basicstyle=\footnotesize,           	% the size of the fonts that are used for the code
  numbers=none,                   		% where to put the line-numbers (none, left,right)
  numberstyle=\tiny\color{gray},  	% the style that is used for the line-numbers
  stepnumber=2,                   		% the step between two line-numbers. If it's 1, each line 
                                  			% will be numbered
  numbersep=5pt,                  		% how far the line-numbers are from the code
  backgroundcolor=\color{white},      	% choose the background color. You must add \usepackage{color}
  showspaces=false,               		% show spaces adding particular underscores
  showstringspaces=false,         		% underline spaces within strings
  showtabs=false,                 		% show tabs within strings adding particular underscores
  frame=single,                   		% adds a frame around the code
  rulecolor=\color{black},        		% if not set, the frame-color may be changed on line-breaks within not-black text (e.g. commens (green here))
  tabsize=2,                      			% sets default tabsize to 2 spaces
  captionpos=b,                   		% sets the caption-position to bottom
  breaklines=true,                		% sets automatic line breaking
  breakatwhitespace=false,        	% sets if automatic breaks should only happen at whitespace
  title=\lstname,                   		% show the filename of files included with \lstinputlisting;
                                  			% also try caption instead of title
  keywordstyle=\color{blue},          	% keyword style
  commentstyle=\color{dkgreen},       % comment style
  stringstyle=\color{mauve},         	% string literal style
  escapeinside={\%*}{*)},            	% if you want to add LaTeX within your code
  morekeywords={*,...}              		% if you want to add more keywords to the set
}

	\renewcommand{\contentsname}{目\qquad 录}
	\tableofcontents
	\section{环境说明}
	\subsection {主机信息}
	\begin{table}[!htp]
	\begin{center}
	\begin{tabular}{|c|c|c|c|c|c|c|c|c|c|c|c|c|}
	\hline
	
	\multirow{2}{*}{主机信息} &  
	\multicolumn{2}{c|}{内网} &
	\multicolumn{2}{c|}{10.1.248.102} &
	\multirow{2}{*}{登陆方式} &
	\multicolumn{2}{c|}{ssh} &
	{用户名} &  {root} & {密码} & {tmxkrcc0} \\
	\cline{2-5}
	\cline{7-12}	
	\multirow{2}{*}{} &  
	\multicolumn{2}{c|}{外网} &
	\multicolumn{2}{c|}{124.207.3.53} &
	\multirow{2}{*}{} &
	\multicolumn{2}{c|}{-} &
	{-} & {-} &	{-} &	{-} \\
	\hline
	{数据库信息} &
	\multicolumn{2}{c|}{IP} & 
	\multicolumn{2}{c|}{10.1.248.102}  &
	{端口} & 
	\multicolumn{2}{c|}{默认} &
	{用户名} & {svnmanager} &
	{密码} & {111111} \\
	\hline
	\end{tabular}
	\end{center}
	\end{table}
	
	\subsection {主要配置文件信息}
	\begin{table}[!htp]
	\begin{center}
	\begin{tabular}{|c|c|c|c|c|c|c|c|c|c|c|c|c|}
	\hline
	{HTTPD配置文件} &
	\multicolumn{7}{c|}{/etc/httpd/conf/httpd.conf} &
	\multicolumn{4}{c|}{配置SVN仓库信息} \\
	\hline
	{SVN权限配置文件} &
	\multicolumn{7}{c|}{/app/svn/accessfile} &
	\multicolumn{4}{c|}{配置SVN的组关联关系} \\
	\hline
	\end{tabular}
	\end{center}
	\end{table}
	
	\subsection {数据库信息}
	\begin{table}[!htp]
	\begin{center}
	\begin{tabular}{|c|c|c|c|c|c|c|c|c|c|c|c|c|}
	\hline
	{数据库信息} &
	\multicolumn{2}{c|}{IP} & 
	\multicolumn{2}{c|}{10.1.248.102}  &
	{端口} & 
	\multicolumn{2}{c|}{默认} &
	{用户名} & {svnmanager} &
	{密码} & {111111} \\
	\hline
	\end{tabular}
	\end{center}
	\end{table}
	
	\subsection {登陆信息}
	\begin{table}[!htp]
	\begin{center}
	\begin{tabular}{|c|c|c|c|c|c|c|c|c|c|c|c|c|}
	\hline
	\multicolumn{11}{|c|}{SVN仓库信息} \\
	\hline
	\multicolumn{2}{|c|}{页面} & 
	\multicolumn{5}{c|}{http://10.1.248.101:8089/svnmanager}  &
	{用户名} & {admin} &
	{密码} & {123456} \\
	\hline
	\end{tabular}
	\end{center}
	\end{table}
	
%\begin{tabular}{|c|c|c|c|c|}
%\hline
%\multirow{2}{*}{Multi-Row} &
%\multicolumn{2}{c|}{Multi-Column} &
%\multicolumn{2}{c|}{\multirow{2}{*}{Multi-Row and Col}} \\
%\cline{2-3}
%   & column-1 & column-2 & \multicolumn{2}{c|}{} \\
%\hline
%label-1 & label-2 & label-3 & label-4 & label-5 \\
%\hline
%\end{tabular}

	\section{操作说明}
	\subsection {新建仓库}
	\subsubsection {页面添加} 
		\begin{enumerate}[step 1]
		\item 登陆http://10.1.248.101:8089/svnmanager
		\item 配置库管理->创建新库
		\item 输入库名信息,确认
		\end{enumerate}
	\subsubsection {配置文件添加} 
		\begin{enumerate}[step 1]
		\item 登陆SVN仓库主机
		\item 在配置文件httpd.conf中添加一个Location
		\begin{lstlisting}[language={[ANSI]C}] {}
<Location /Uniprm>
DAV svn
# SVNListParentPath on
SVNPath /app/svn/Uniprm
AuthType Basic
AuthName "ZGSM SVN"
AuthUserFile /app/svn/passwdfile
AuthzSVNAccessFile /app/svn/accessfile
Require valid-user
</Location>
		\end{lstlisting}
		\item 为仓库配置组权限控制 在accessfile中修改成如下信息, 
		\footnote{@部分组名与建立的组名一致,查看文件开头部分}
		\begin{lstlisting} {}
[rd3:/]
admin=rw
@RD3 读写组=rw
@RD3 只读组=r  
		\end{lstlisting}

		\item 重启apache. 
		\begin{lstlisting}[language={[ANSI]C}] {}
service httpd restart
		\end{lstlisting}
		\end{enumerate}
		
	\subsection {新建组}
	\subsubsection{页面添加}
	无

	\subsection {新建人员}
	\subsubsection{页面添加}
	无

	\subsection {特殊事项}
	\subsubsection{针对svnmanager中组添加人员的bug}
		\begin{enumerate}[step 1]
		\item 登陆mysql数据库
		\item 查询需要添加的用户ID
		\begin{lstlisting}[language={SQL}] {}
select * from users where name in ('');
		\end{lstlisting}
		\item 查询需要添加的组ID
		\begin{lstlisting}[language={SQL}]{}
select * from groups where name = '';
		\end{lstlisting}
		\item 将信息插入到usergroups中
		\begin{lstlisting}[language={SQL}]{}
insert into usersgroups values(user_id, group_id); 
		\end{lstlisting}
		\item 登陆http://10.1.248.101:8089/svnmanager
		\item 组管理->编辑组->选择需要变更的组
		\item {\color{red}{不要选择任何人员,点击确认}}
		\item 添加一个用户,之后删除 \footnote{这是一个bug,只有当用户删除的时候,信息才会被同步}
		\end{enumerate}	
	
	\subsubsection {简化批量文件时的一些小操作}
		\begin{enumerate}[step 1]
		\item Shell操作
将用户名存放在文件中,每行一个用户名之后使用以下命令
		\begin{lstlisting}[language={[ANSI]C}]{}
sed -e 's/ //g' username | cat | xargs -n10000 | sed -e 's/ /#,#/g' | sed -e 's/^/#/' | sed -e 's/$/#/' | sed -e "s/#/'/g"
		\end{lstlisting}
		参考指令:  
		\begin{table}[!htp]
		\begin{center}
		\begin{tabular}{|c|c|c|c|c|clc|}
		\hline
序号 & 工号 & 姓名 & NT账号 & 账号 & 
		\multicolumn{2}{|c|}{密码}  \\
		\hline
1  & 66857 & 周园 & zhouyuan2 & zhouyuan2  & 
		\multicolumn{2}{|c|}{123456}  \\
		\hline
		\end{tabular}
		\end{center}
		\end{table}
		
		\begin{lstlisting}[language={[ANSI]C}]{}
针对上表这种格式 awk '{print $4}' execluserfile 可替换上语句中的sed -e 's/ //g' username 部分
		\end{lstlisting}
		\item 数据库操作
		\begin{lstlisting}[language={[ANSI]C}]{}
建立一张表users_insert_list 存放ID字段,状态.
		\end{lstlisting}
		\item 使用一下sql将用户写入到新建的表中 name部分用之前的信息替代
		\begin{lstlisting}[language={SQL}]{}
insert into users_insert_list select id, 0 from users where name in ('name'); 
		\end{lstlisting}
		\item 获取到组ID, 
		\item 使用以下SQL讲用户记入配置表
		\begin{lstlisting}[language={SQL}]{}
insert into usersgroups select id, 'groupid' from users u, users_insert_list l where u.id = l.id 
		\end{lstlisting}
		\end{enumerate}

	\section{CVS}
	\subsection{cvs添加用户}
		\subsubsection{prm}
		\begin{lstlisting}[language={C}]{}
ssh root@10.1.248.102
cd /bss/cvsroot/prm/CVSROOT
        脚本名 用户  密码  邮箱 是否指定密码
perl mail.pl user password email@email.com 0
不提示任何信息则表示添加成功,并发送邮件到邮箱
提示信息 表示存在用户
grep user newuser.log 最后一次出现的记录为当前密码
		\end{lstlisting}

		\subsubsection{/bss/atc}
		\begin{lstlisting}[language={C}]{}
ssh root@10.1.248.102
cd /bss/atc/CVSROOT
perl mail.pl user password email@email.com 0
		\end{lstlisting}

	\section{其他}
	\subsection{WINDOWS系统相关}
		\subsubsection{清除CVS目录}
		\begin{lstlisting}[language={C}]{}
echo prj_home
set prj_home=E:\workspaces\WS_Shine\Shine
@for /r %prj_home%\deployed_metas %%a in (.) do @if exist "%%a\CVS" rd /s /q "%%a\CVS"
@echo 文件清除结束.
@pause
		\end{lstlisting}
		
		\subsubsection{CVS复制代码仓库}
		\begin{lstlisting}[language={C}]{}
cvs import -m "CU_4G" CU_4G HEAD start
		\end{lstlisting}

\end{document}